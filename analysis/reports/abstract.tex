Within the field of political science, interest in statistical methods for the automatic analysis of text has grown considerably in recent years: see, for example, Grimmer 2011, Cormack, etc., etc. (REFS)

At the same time, interest in the analysis of social networks has been growing. (DREW REVIEW)

In our recent work, we have collected a large sample of Twitter data, which provides a rich data set with a large quantity of text and a social graph. We compare the usefulness of text analysis and social network analysis for predicting the political views of Twitter users for which we do not have roll call data.

Here we demonstrate how these methods can be combined to provide a new method for measuring political affiliation and dealing with the new types of observational data that are becoming available. Twitter, the very popular web service, provides a large scale social network in which connections are not reciprocated (unlike Facebook's bidirectional friend relationship); for that reason, Twitter provides one of the most easily accessible directed social graphs in the world. In addition, the primary purpose of Twitter is to distribute small bits of text (no more than 140 characters) to all of the people who have signed up to follow your updates. Thus Twitter also provides us with a new source of data for text analysis.

To demonstrate how Twitter provides, by itself, sufficient materials for a large scale analysis of political affiliations and expression in the U.S., we take the adjacency graph for the members of the senate and the house. Following (Aaron?) King 2011, we compute the principal components of the adjacency graph and use it to estimate a form of ideal points. 

We also run the same style of text analysis using ideal points generated using roll call data in the style of Simon Jackman. We compare results between the two sets of analyses, focusing on commonalities in terms selected by both analyses.

TIME SERIES OF TWITTER ACCOUNTS OPENING?

COMPARISON OF TWO TYPE OF IDEAL POINTS

EXPLANATION OF LASSO VS. RIDGE: EXAMPLES OF TOP WEIGHTED WORDS

RAW COUNTS, LOG1P COUNTS, SCALE COUNTS, LOG1P + SCALE COUNTS

We have provided a proof of concept example of the viability of combining social network analysis and text analysis to mine Twitter for insight into the political culture of the U.S.
